% Options for packages loaded elsewhere
\PassOptionsToPackage{unicode}{hyperref}
\PassOptionsToPackage{hyphens}{url}
%
\documentclass[
  ignorenonframetext,
  aspectratio=43,
]{beamer}
\usepackage{pgfpages}
\setbeamertemplate{caption}[numbered]
\setbeamertemplate{caption label separator}{: }
\setbeamercolor{caption name}{fg=normal text.fg}
\beamertemplatenavigationsymbolsempty
% Prevent slide breaks in the middle of a paragraph
\widowpenalties 1 10000
\raggedbottom
\setbeamertemplate{part page}{
  \centering
  \begin{beamercolorbox}[sep=16pt,center]{part title}
    \usebeamerfont{part title}\insertpart\par
  \end{beamercolorbox}
}
\setbeamertemplate{section page}{
  \centering
  \begin{beamercolorbox}[sep=12pt,center]{part title}
    \usebeamerfont{section title}\insertsection\par
  \end{beamercolorbox}
}
\setbeamertemplate{subsection page}{
  \centering
  \begin{beamercolorbox}[sep=8pt,center]{part title}
    \usebeamerfont{subsection title}\insertsubsection\par
  \end{beamercolorbox}
}
\AtBeginPart{
  \frame{\partpage}
}
\AtBeginSection{
  \ifbibliography
  \else
    \frame{\sectionpage}
  \fi
}
\AtBeginSubsection{
  \frame{\subsectionpage}
}

\usepackage{amsmath,amssymb}
\usepackage{iftex}
\ifPDFTeX
  \usepackage[T1]{fontenc}
  \usepackage[utf8]{inputenc}
  \usepackage{textcomp} % provide euro and other symbols
\else % if luatex or xetex
  \usepackage{unicode-math}
  \defaultfontfeatures{Scale=MatchLowercase}
  \defaultfontfeatures[\rmfamily]{Ligatures=TeX,Scale=1}
\fi
\usepackage{lmodern}
\usetheme[]{Berlin}
\ifPDFTeX\else  
    % xetex/luatex font selection
\fi
% Use upquote if available, for straight quotes in verbatim environments
\IfFileExists{upquote.sty}{\usepackage{upquote}}{}
\IfFileExists{microtype.sty}{% use microtype if available
  \usepackage[]{microtype}
  \UseMicrotypeSet[protrusion]{basicmath} % disable protrusion for tt fonts
}{}
\makeatletter
\@ifundefined{KOMAClassName}{% if non-KOMA class
  \IfFileExists{parskip.sty}{%
    \usepackage{parskip}
  }{% else
    \setlength{\parindent}{0pt}
    \setlength{\parskip}{6pt plus 2pt minus 1pt}}
}{% if KOMA class
  \KOMAoptions{parskip=half}}
\makeatother
\usepackage{xcolor}
\newif\ifbibliography
\setlength{\emergencystretch}{3em} % prevent overfull lines
\setcounter{secnumdepth}{-\maxdimen} % remove section numbering

\usepackage{color}
\usepackage{fancyvrb}
\newcommand{\VerbBar}{|}
\newcommand{\VERB}{\Verb[commandchars=\\\{\}]}
\DefineVerbatimEnvironment{Highlighting}{Verbatim}{commandchars=\\\{\}}
% Add ',fontsize=\small' for more characters per line
\usepackage{framed}
\definecolor{shadecolor}{RGB}{241,243,245}
\newenvironment{Shaded}{\begin{snugshade}}{\end{snugshade}}
\newcommand{\AlertTok}[1]{\textcolor[rgb]{0.68,0.00,0.00}{#1}}
\newcommand{\AnnotationTok}[1]{\textcolor[rgb]{0.37,0.37,0.37}{#1}}
\newcommand{\AttributeTok}[1]{\textcolor[rgb]{0.40,0.45,0.13}{#1}}
\newcommand{\BaseNTok}[1]{\textcolor[rgb]{0.68,0.00,0.00}{#1}}
\newcommand{\BuiltInTok}[1]{\textcolor[rgb]{0.00,0.23,0.31}{#1}}
\newcommand{\CharTok}[1]{\textcolor[rgb]{0.13,0.47,0.30}{#1}}
\newcommand{\CommentTok}[1]{\textcolor[rgb]{0.37,0.37,0.37}{#1}}
\newcommand{\CommentVarTok}[1]{\textcolor[rgb]{0.37,0.37,0.37}{\textit{#1}}}
\newcommand{\ConstantTok}[1]{\textcolor[rgb]{0.56,0.35,0.01}{#1}}
\newcommand{\ControlFlowTok}[1]{\textcolor[rgb]{0.00,0.23,0.31}{#1}}
\newcommand{\DataTypeTok}[1]{\textcolor[rgb]{0.68,0.00,0.00}{#1}}
\newcommand{\DecValTok}[1]{\textcolor[rgb]{0.68,0.00,0.00}{#1}}
\newcommand{\DocumentationTok}[1]{\textcolor[rgb]{0.37,0.37,0.37}{\textit{#1}}}
\newcommand{\ErrorTok}[1]{\textcolor[rgb]{0.68,0.00,0.00}{#1}}
\newcommand{\ExtensionTok}[1]{\textcolor[rgb]{0.00,0.23,0.31}{#1}}
\newcommand{\FloatTok}[1]{\textcolor[rgb]{0.68,0.00,0.00}{#1}}
\newcommand{\FunctionTok}[1]{\textcolor[rgb]{0.28,0.35,0.67}{#1}}
\newcommand{\ImportTok}[1]{\textcolor[rgb]{0.00,0.46,0.62}{#1}}
\newcommand{\InformationTok}[1]{\textcolor[rgb]{0.37,0.37,0.37}{#1}}
\newcommand{\KeywordTok}[1]{\textcolor[rgb]{0.00,0.23,0.31}{#1}}
\newcommand{\NormalTok}[1]{\textcolor[rgb]{0.00,0.23,0.31}{#1}}
\newcommand{\OperatorTok}[1]{\textcolor[rgb]{0.37,0.37,0.37}{#1}}
\newcommand{\OtherTok}[1]{\textcolor[rgb]{0.00,0.23,0.31}{#1}}
\newcommand{\PreprocessorTok}[1]{\textcolor[rgb]{0.68,0.00,0.00}{#1}}
\newcommand{\RegionMarkerTok}[1]{\textcolor[rgb]{0.00,0.23,0.31}{#1}}
\newcommand{\SpecialCharTok}[1]{\textcolor[rgb]{0.37,0.37,0.37}{#1}}
\newcommand{\SpecialStringTok}[1]{\textcolor[rgb]{0.13,0.47,0.30}{#1}}
\newcommand{\StringTok}[1]{\textcolor[rgb]{0.13,0.47,0.30}{#1}}
\newcommand{\VariableTok}[1]{\textcolor[rgb]{0.07,0.07,0.07}{#1}}
\newcommand{\VerbatimStringTok}[1]{\textcolor[rgb]{0.13,0.47,0.30}{#1}}
\newcommand{\WarningTok}[1]{\textcolor[rgb]{0.37,0.37,0.37}{\textit{#1}}}

\providecommand{\tightlist}{%
  \setlength{\itemsep}{0pt}\setlength{\parskip}{0pt}}\usepackage{longtable,booktabs,array}
\usepackage{calc} % for calculating minipage widths
\usepackage{caption}
% Make caption package work with longtable
\makeatletter
\def\fnum@table{\tablename~\thetable}
\makeatother
\usepackage{graphicx}
\makeatletter
\def\maxwidth{\ifdim\Gin@nat@width>\linewidth\linewidth\else\Gin@nat@width\fi}
\def\maxheight{\ifdim\Gin@nat@height>\textheight\textheight\else\Gin@nat@height\fi}
\makeatother
% Scale images if necessary, so that they will not overflow the page
% margins by default, and it is still possible to overwrite the defaults
% using explicit options in \includegraphics[width, height, ...]{}
\setkeys{Gin}{width=\maxwidth,height=\maxheight,keepaspectratio}
% Set default figure placement to htbp
\makeatletter
\def\fps@figure{htbp}
\makeatother
% definitions for citeproc citations
\NewDocumentCommand\citeproctext{}{}
\NewDocumentCommand\citeproc{mm}{%
  \begingroup\def\citeproctext{#2}\cite{#1}\endgroup}
\makeatletter
 % allow citations to break across lines
 \let\@cite@ofmt\@firstofone
 % avoid brackets around text for \cite:
 \def\@biblabel#1{}
 \def\@cite#1#2{{#1\if@tempswa , #2\fi}}
\makeatother
\newlength{\cslhangindent}
\setlength{\cslhangindent}{1.5em}
\newlength{\csllabelwidth}
\setlength{\csllabelwidth}{3em}
\newenvironment{CSLReferences}[2] % #1 hanging-indent, #2 entry-spacing
 {\begin{list}{}{%
  \setlength{\itemindent}{0pt}
  \setlength{\leftmargin}{0pt}
  \setlength{\parsep}{0pt}
  % turn on hanging indent if param 1 is 1
  \ifodd #1
   \setlength{\leftmargin}{\cslhangindent}
   \setlength{\itemindent}{-1\cslhangindent}
  \fi
  % set entry spacing
  \setlength{\itemsep}{#2\baselineskip}}}
 {\end{list}}
\usepackage{calc}
\newcommand{\CSLBlock}[1]{\hfill\break\parbox[t]{\linewidth}{\strut\ignorespaces#1\strut}}
\newcommand{\CSLLeftMargin}[1]{\parbox[t]{\csllabelwidth}{\strut#1\strut}}
\newcommand{\CSLRightInline}[1]{\parbox[t]{\linewidth - \csllabelwidth}{\strut#1\strut}}
\newcommand{\CSLIndent}[1]{\hspace{\cslhangindent}#1}

\titlegraphic{\includegraphics[width = 3cm]{xjtluicon.jpg}}
\makeatletter
\@ifpackageloaded{caption}{}{\usepackage{caption}}
\AtBeginDocument{%
\ifdefined\contentsname
  \renewcommand*\contentsname{Table of contents}
\else
  \newcommand\contentsname{Table of contents}
\fi
\ifdefined\listfigurename
  \renewcommand*\listfigurename{List of Figures}
\else
  \newcommand\listfigurename{List of Figures}
\fi
\ifdefined\listtablename
  \renewcommand*\listtablename{List of Tables}
\else
  \newcommand\listtablename{List of Tables}
\fi
\ifdefined\figurename
  \renewcommand*\figurename{Figure}
\else
  \newcommand\figurename{Figure}
\fi
\ifdefined\tablename
  \renewcommand*\tablename{Table}
\else
  \newcommand\tablename{Table}
\fi
}
\@ifpackageloaded{float}{}{\usepackage{float}}
\floatstyle{ruled}
\@ifundefined{c@chapter}{\newfloat{codelisting}{h}{lop}}{\newfloat{codelisting}{h}{lop}[chapter]}
\floatname{codelisting}{Listing}
\newcommand*\listoflistings{\listof{codelisting}{List of Listings}}
\makeatother
\makeatletter
\makeatother
\makeatletter
\@ifpackageloaded{caption}{}{\usepackage{caption}}
\@ifpackageloaded{subcaption}{}{\usepackage{subcaption}}
\makeatother
\ifLuaTeX
  \usepackage{selnolig}  % disable illegal ligatures
\fi
\usepackage{bookmark}

\IfFileExists{xurl.sty}{\usepackage{xurl}}{} % add URL line breaks if available
\urlstyle{same} % disable monospaced font for URLs
\hypersetup{
  pdftitle={Basic Probabilities, Sampling Distribution},
  pdfauthor={Mu He},
  hidelinks,
  pdfcreator={LaTeX via pandoc}}

\title{Basic Probabilities, Sampling Distribution}
\author{Mu He}
\date{}
\institute{Xi'an Jiaotong-Liverpool University}

\begin{document}
\frame{\titlepage}

\renewcommand*\contentsname{Table of contents}
\begin{frame}[allowframebreaks]
  \frametitle{Table of contents}
  \tableofcontents[hideallsubsections]
\end{frame}
\section{Basic Probabilities}\label{basic-probabilities}

\begin{frame}{Review}
\phantomsection\label{review}
The notes are based on {[}1{]} {[}2{]} {[}3{]} {[}4{]} {[}5{]}

\begin{block}{Expectation}
\phantomsection\label{expectation}
\begin{quote}
Expectation for Discrete Random Variable: \[
E(X)=\sum x_i f(x_i)
\]
\end{quote}

\begin{quote}
Expectation for Continuous Random Variable: \[
E(X)=\int x f(x) dx
\]
\end{quote}
\end{block}
\end{frame}

\section{Convergence of Sequence}\label{convergence-of-sequence}

\begin{frame}{Types of Convergence}
\phantomsection\label{types-of-convergence}
In this section, we will develop the theoretical background to study the
convergence of a sequence of random variables in more detail. In
particular, we will define different types of convergence. When we say
that the sequence \(X_n\) converges to \(X\), it means that \(X_n\) `s
are getting'`closer and closer'' to \(X\). Different types of
convergence refer to different ways of defining what `'closer'' means.
We also discuss how different types of convergence are related.
\end{frame}

\begin{frame}{Types of Convergence}
\phantomsection\label{types-of-convergence-1}
\begin{block}{Convergence of Sequence}
\phantomsection\label{convergence-of-sequence-1}
\begin{quote}
Convergence of Sequence: A sequence \(a_1,a_2,a_3, \cdots, a_n\)
converges to a limit \(L\) if \[
\lim_{n\rightarrow \infty} a_n=L
\] That is, for any \(\epsilon>0\), there exists an \(N\in \mathbb{N}\)
such that \[
|a_n-L|<\epsilon, \quad \text{ for all } n> N
\]
\end{quote}
\end{block}
\end{frame}

\begin{frame}{Basic Probability Theory}
\phantomsection\label{basic-probability-theory}
\begin{block}{\(\sigma\)-algrbra, algebra and semi-ring}
\phantomsection\label{sigma-algrbra-algebra-and-semi-ring}
\begin{quote}
Definition: algebra, \(\sigma\)-algrbra, semi-ring
\end{quote}

FAQ:
\href{https://math.stackexchange.com/questions/233702/example-of-an-algebra-which-is-not-a-\%CF\%83-algebra}{Exampleof
an algebra not a \(\sigma\)-algebra}

Not required, just as an introduction. If you are very intersted in
probability theory, it is a recommendation to find out why we need these
definitions.
\end{block}
\end{frame}

\begin{frame}{Convergence of Sequence}
\phantomsection\label{convergence-of-sequence-2}
\begin{block}{Sequence of Random Variables}
\phantomsection\label{sequence-of-random-variables}
In statistics, we draw a sample to make inference of the population,
then, if we repeatly draw samples, we will have a sequence of samples
from the same population, we usually refer them as i.i.d. (independent
and identical distributed) or random samples. This can be denoted as

\[
\{\Omega,\Sigma,P\}
\]

where \(\Omega\) is the sample space.
\end{block}
\end{frame}

\begin{frame}{Convergence of Sequence}
\phantomsection\label{convergence-of-sequence-3}
\begin{block}{Sequence of Random Variables}
\phantomsection\label{sequence-of-random-variables-1}
\[
\Omega=\{\omega_1,\omega_2,\cdots, \omega_n\}, \quad w_i \text{ are simple(single) events}
\]

\(\Sigma\) is the \(\sigma\)-algebra (You may consider it is set of the
sets of simple events in brief) and \(P\) is a probability measure.
\end{block}
\end{frame}

\begin{frame}{Convergence of Sequence}
\phantomsection\label{convergence-of-sequence-4}
However, if we consider the samples not necessarily from the same
population, we may have a sequence of random variables
\(X_1,X_2,\cdots\), and an correspnded underlying sample space
\(\Omega\). In particular, each \(X_n\) is a function from its
\(\Omega\), to real numbers through the probability measure \(P\).

In other words, a sequence of random variables is in fact a sequence of
functions (Mapping, or \(P\), or a probability measure)
\(X_n:\Omega\rightarrow \mathbb{R}\) , such as

\[
P(\omega_i)=x_i, \quad \omega_i \in \Omega \text{ and } \sum x_i=1, \quad i = 1,\cdots,n
\]
\end{frame}

\begin{frame}{Example: Convergence of Sequence of R.V.}
\phantomsection\label{example-convergence-of-sequence-of-r.v.}
Consider the following random experiment: A fair coin is tossed once.
Here, the sample space has only two elements \(S=\{H,T\}\). We define a
sequence of random variables \(X_1,X_2,\cdots\) on this sample space as
follows:

\[
 X_n(s) = \left\{
\begin{array}{l l}
\frac{1}{n+1} & \qquad \text{ if }s=H \\
& \qquad \\
1 & \qquad \text{ if }s=T
\end{array} \right.
\]

\begin{itemize}
\item
  Are the \(X_i\) independent?

  No, they are dependent as they are measuring the same coin.
\end{itemize}
\end{frame}

\begin{frame}{Example: Convergence of Sequence of R.V.}
\phantomsection\label{example-convergence-of-sequence-of-r.v.-1}
\begin{itemize}
\tightlist
\item
  Find the PMF and CDF of \(X_n\), \(F_{X_n}(x)\) for
  \(n=1,2,3,\cdots\).
\end{itemize}

The PMF are

\[
P_{{\large X_n}}(x)=P(X_n=x) = \left\{
   \begin{array}{l l}
   \frac{1}{2} & \qquad \textrm{ if }x=\frac{1}{n+1} \\
   & \qquad \\
   \frac{1}{2} & \qquad \textrm{ if }x=1
   \end{array} \right.
\] Correspondingly, the CDF are

\[
F_{{\large X_n}}(x)=P(X_n \leq x) = \left\{
\begin{array}{l l}
   1 & \qquad \textrm{ if }x \geq 1\\
   & \qquad \\
   \frac{1}{2} & \qquad \textrm{ if }\frac{1}{n+1} \leq x <1 \\
   & \qquad \\
   0 & \qquad \textrm{ if }x< \frac{1}{n+1}
   \end{array} \right.
\]
\end{frame}

\begin{frame}{Example: Convergence of Sequence of R.V.}
\phantomsection\label{example-convergence-of-sequence-of-r.v.-2}
\begin{itemize}
\tightlist
\item
  As \(n\) goes to infinity, what does \(F_{X_n}(x)\) look like?
\end{itemize}
\end{frame}

\begin{frame}{Example: Suppose that \(T\) is R.V. as above, derive its
p.d.f.}
\phantomsection\label{example-suppose-that-t-is-r.v.-as-above-derive-its-p.d.f.}
\begin{enumerate}
\item
  If \(T\) is given by \(\frac{U}{\sqrt{V/k}}\), find the joint density
  of \(U\) and \(V\).
\item
  Find the density function of \(T\).
\end{enumerate}

\[
f_{U,V}(u,v) = \underbrace{\frac{1}{(2\pi)^{1/2}} e^{-u^2/2}}_{\text{pdf } N(0,1)}\quad \underbrace{\frac{1}{\Gamma(\frac{k}{2})\,2^{k/2}}\,v^{(k/2)-1}\, e^{-v/2}}_{\text{pdf }\chi^2_k}
\]

Denote

\[
t=\frac{u}{\sqrt{v/k}}, \quad w=v
\]

where

\[
u=t(\frac{w}{k})^{1/2}, \quad v= w
\]

The Jacobian matrix can be find as

\[
J=\begin{vmatrix}
   \frac{du}{dt} & \frac{du}{dw}\\
   \frac{dv}{dt} & \frac{dv}{dw}\\
   \end{vmatrix}=\begin{vmatrix}
   (\frac{w}{k})^{1/2} & \frac{1}{2}t(\frac{1}{wk})^{1/2}\\
   0&1
   \end{vmatrix}=(\frac{w}{k})^{1/2}
\]
\end{frame}

\begin{frame}[fragile]{Example: Calculation}
\phantomsection\label{example-calculation}
The tensile strength for a type of wire is normally distributed with
unknown mean \(\mu\) and unknown variance \(\sigma^2\). Six pieces of
wire were randomly selected from a large roll; \(Y_i\), the tensile
strength for portion \(i\), is measured for \(i = 1, 2, . . . , 6\). The
population mean \(\mu\) and variance \(\sigma^2\) can be estimated by
\(\bar{Y}\) and \(s^2\), respectively.

Find the approximate probability that \(\bar{Y}\) will be within
\(2S/\sqrt{n}\) of the true population mean \(\mu\).

\begin{verbatim}
[1] 0.8980605
\end{verbatim}
\end{frame}

\begin{frame}[fragile]{Example: Calculation}
\phantomsection\label{example-calculation-1}
As

\[
T=\frac{\bar{Y}-\mu}{S/\sqrt{n}}
\]

Then

\[
P(|\bar{Y}-\mu|\leq 2S/\sqrt{n})=P(-2\leq T \leq 2)= P(T\leq 2)-P(T\leq -2)=?
\]

\begin{Shaded}
\begin{Highlighting}[]
\FunctionTok{pt}\NormalTok{(}\DecValTok{2}\NormalTok{,}\DecValTok{5}\NormalTok{)}\SpecialCharTok{{-}}\FunctionTok{pt}\NormalTok{(}\SpecialCharTok{{-}}\DecValTok{2}\NormalTok{,}\DecValTok{5}\NormalTok{)}
\end{Highlighting}
\end{Shaded}

\begin{verbatim}
[1] 0.8980605
\end{verbatim}
\end{frame}

\begin{frame}{The F Distribution}
\phantomsection\label{the-f-distribution}
Suppose that we want to compare the variances of two normal populations
based on information contained in independent random samples from the
two populations.

\begin{quote}
The F Distribution: Let \(W_1\) and \(W_2\) be independent \(\chi^2\)
distributed random variables with \(v_1\) and \(v_2\) degree of freedom.
Then, \[
F=\frac{W_1/v_1}{W_2/v_2}=\frac{(n-1)S^2_1/\sigma^2_1/(n_1-1)}{(n-1)S^2_2/\sigma^2_2/(n_2-1)}=\frac{S^2_1/\sigma^2_1}{S^2_2/\sigma^2_2}
\] is an F distribution, \(F(v_1=n_1-1,v_2=n_2-1)\).
\end{quote}
\end{frame}

\begin{frame}[fragile]{Example: Calculation}
\phantomsection\label{example-calculation-2}
If there are two popluation with equal variance, we draw two sample with
size \(n_1=6\) and \(n_2=10\), such that

\[
P(\frac{S^2_1}{S^2_2} \leq b)=0.95
\]

What is the value of b?

\begin{verbatim}
[1] 3.481659
\end{verbatim}
\end{frame}

\begin{frame}{Reference}
\phantomsection\label{reference}
\phantomsection\label{refs}
\begin{CSLReferences}{0}{1}
\bibitem[\citeproctext]{ref-pishro-nik2014}
\CSLLeftMargin{{[}1{]} }%
\CSLRightInline{\textsc{Pishro-Nik}, H. (2014). \emph{Introduction to
probability, statistics, and random processes}. Kappa Research, LLC Blue
Bell, PA, USA.}

\bibitem[\citeproctext]{ref-larsen2005}
\CSLLeftMargin{{[}2{]} }%
\CSLRightInline{\textsc{Larsen}, R. J. and \textsc{Marx}, M. L. (2005).
\emph{An introduction to mathematical statistics}. Prentice Hall
Hoboken, NJ.}

\bibitem[\citeproctext]{ref-casella2024}
\CSLLeftMargin{{[}3{]} }%
\CSLRightInline{\textsc{Casella}, G. and \textsc{Berger}, R. (2024).
\emph{Statistical inference}. CRC Press.}

\bibitem[\citeproctext]{ref-tao}
\CSLLeftMargin{{[}4{]} }%
\CSLRightInline{\textsc{Tao}, T. (2008).
\href{https://terrytao.wordpress.com/2008/06/18/the-strong-law-of-large-numbers/}{The
strong law of large number}.}

\bibitem[\citeproctext]{ref-derivati}
\CSLLeftMargin{{[}5{]} }%
\CSLRightInline{\textsc{Anon}. (2014).
\href{https://math.stackexchange.com/questions/474733/derivation-of-the-density-function-of-student-t-distribution-from-this-big-integ}{Derivation
of t distribution density}.}

\end{CSLReferences}
\end{frame}



\end{document}
